\documentclass{article}
\usepackage{lmodern} % Fix the font size warnings
\usepackage{fancyhdr} % Required for custom headers
\usepackage{lastpage} % Required to determine the last page for the footer
\usepackage{extramarks} % Required for headers and footers
\usepackage{graphicx} % Required to insert images
\usepackage{mathabx} % Required for hash sign
\usepackage{xfrac} % Nice fractions

\usepackage{multicol}

% Margins
\topmargin=-0.5in
\evensidemargin=0in
\oddsidemargin=-0.5in
\textwidth=7.5in
\textheight=9.0in
\headsep=0.25in 

% paragraphs
\usepackage{parskip}

\pagestyle{fancy}

\rhead{Desserts and sweets}
\lhead{\textbf{Snickerdoodles}}
\chead{}
\title{Snickerdoodles}

\begin{document}

Like many of our recipes, this one started off in Cook's Illustrated Baking, but\dots it really
ended up being ours after several iterations. You see, Dad can't stop messing with recipes and will
keep tweaking them until they're \textit(just right).

What makes this a better snickerdoodle is the addition of cinnamon and allspice to the
dry ingredients in addition to that in which the cookies are rolled.

\textbf{Ingredients}

\begin{multicols}{2}
    \begin{itemize}
        \item 2\sfrac{1}{4} cups all purpose flour
        \item 2 teapoons cream of tartar
        \item 1 teaspoon baking soda
        \item \sfrac{1}{2} teapoon salt
        \item 2 teaspoon cinnamon
        \item \sfrac{1}{4} teapoon allspice

        \item 1\sfrac{1}{2} sticks butter (12 tablespoons), softened slightly
        \item \sfrac{1}{4} vegetable shortening
        \item 1\sfrac{1}{2} cup granulated sugar, plus 3 tablespoons for rolling
        \item 2 large egg
        \item 1 tablespoons cinnamon for rolling
    \end{itemize}
\end{multicols}

\textbf{Directions}

\begin{enumerate}
    \item Preheat oven to 400 degrees.
    \item Whisk the flour, cream of tartar, baking soda, salt, allspice, and 2 teapoons cinnamon
          in a small bowl.
    \item Cream the butter, shortening, and sugar together until light and fluffy. You want to make sure you get
          enough air into the mixture and cut all that butter nicely.
    \item Add the vanilla and the eggs, one egg at a time. Beat the wet ingredients until quite smooth.
    \item Add the dry ingredients to the mixer and mix well.
    \item Using a small $\varhash 20$ scoop spoon the dough onto parchment-lined cookie sheets.
    \item Bake until the edges of the cookies are beginning to set and the centers are soft
          and puffy, about 9 to 11 minutes, rotating the baking sheets top to bottom and front
          to back halfway through baking. Let the cookies cool on the baking sheets for 2 to 3
          minutes then move to a wire rack.
\end{enumerate}

\medskip

% I suppose pictures could go here

\end{document}
