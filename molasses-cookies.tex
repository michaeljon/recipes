\documentclass{article}
\usepackage{lmodern} % Fix the font size warnings
\usepackage{fancyhdr} % Required for custom headers
\usepackage{lastpage} % Required to determine the last page for the footer
\usepackage{extramarks} % Required for headers and footers
\usepackage{graphicx} % Required to insert images
\usepackage{xfrac} % Nice fractions
\usepackage{amsmath}

\usepackage{multicol}

% Margins
\topmargin=-0.5in
\evensidemargin=0in
\oddsidemargin=-0.5in
\textwidth=7.5in
\textheight=9.0in
\headsep=0.25in 

% paragraphs
\usepackage{parskip}

\pagestyle{fancy}

\rhead{Desserts and sweets}
\lhead{\textbf{Molasses Cookies}}
\chead{}
\title{Molasses Cookies}

\begin{document}
Ok, this is one of those recipes that started out somewhere else, and
over time, has been changed enough to be considered mine now. Just in
case you're wondering, this is also one of those recipes that converts
peopleinto molasses cookie fans. Even the ones who say they don't like
molasses cookies. Just sayin.

Makes 22 cookies. Yeah, don't ask. That's how many it makes. So, I almost
always double this recipe.

For the best flavor, make sure that your spices are fresh. Light or mild
molasses gives the cookies a milder flavor; for a stronger flavor, use
dark or blackstrap molasses (this is what I use). Either way, measure
the molasses in a liquid measuring cup. You can use the five spices to
tune this recipe to your personal taste. I tend to make the first four
``heaping'' sized, but keep the black pepper right around \sfrac{1}{4} teaspoon.

\begin{multicols}{2}

      \textbf{Ingredients}

      \begin{itemize}
            \item 2\sfrac{1}{4} cups (11\sfrac{1}{4} ounces) all purpose flour
            \item 1 teaspoon baking soda
            \item 2 teaspoons ground cinnamon
            \item 2 teaspoons ground ginger
            \item \sfrac{1}{2} to \sfrac{3}{4} teaspoon ground cloves
            \item \sfrac{1}{4} to \sfrac{1}{2} teapoon ground allspice
            \item \sfrac{1}{4} teaspoon black pepper
            \item \sfrac{1}{4} teaspoon salt

                  \columnbreak

            \item 12 tablespoons (1\sfrac{1}{2} sticks) unsalted butter softened but still cool
            \item \sfrac{1}{3} cup packed (2\sfrac{1}{3} ounces) dark brown sugar
            \item \sfrac{1}{3} cup (2\sfrac{1}{3} ounces) granulated sugar
            \item \sfrac{1}{2} cup raw sugar (for rolling)
            \item 1 large egg yolk
            \item 1 teaspoon vanilla
            \item \sfrac{1}{2} cup light or dark molasses (see the note above)
      \end{itemize}

\end{multicols}

\textbf{Directions}

\begin{enumerate}
      \item Adjust an over rack to the middle position and heat the oven to 375 degrees. Line a large baking sheet
            with parchment paper.
      \item Whisk the flour, baking soda, spices, and salt in a medium bowl and set aside.
      \item Cut the butter into tablespoon-sized pieces and beat on medium with the sugars until light and fluffy
            (about three minutes).
      \item Reduce speed to medium-low and add the egg yolk and vanilla; increase the speed to medium and beat
            until incorporated (about 30 seconds). Be sure to scrape the sides of the bowl at least once. Reduce
            speed to medium-low and add the molasses; beat until fully incorporated (about 30 seconds). Scrape the
            bottom and sides of the bowl at least once to avoid globs of butter. Reduce speed to lowest setting
            and add the flour mixture and beat until just incorporated. Once more with the scraping. This will
            take about 30 to 45 seconds.
      \item Use a \#30 cookie dough scoop to form balls and roll them in the raw sugar.
            Place 2 inches apart on the parchment-line baking sheets and bake for 10 minutes. The cookies will
            still look slightly undercooked. That's Ok. Don't try to bake two sheets at the same time unless
            you really trust your oven.
      \item Cool the cookies for 5 minutes on the baking sheet then move to a wire rack. Avoid the rush of people.
\end{enumerate}


% I suppose pictures could go here

\end{document}
