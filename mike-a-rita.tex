\documentclass{article}
\usepackage{lmodern} % Fix the font size warnings
\usepackage{fancyhdr} % Required for custom headers
\usepackage{lastpage} % Required to determine the last page for the footer
\usepackage{extramarks} % Required for headers and footers
\usepackage{graphicx} % Required to insert images
\usepackage{xfrac} % Nice fractions
\usepackage{amsmath}

\usepackage{multicol}

% Margins
\topmargin=-0.5in
\evensidemargin=0in
\oddsidemargin=-0.5in
\textwidth=7.5in
\textheight=9.0in
\headsep=0.25in 

% paragraphs
\usepackage{parskip}

\pagestyle{fancy}

\rhead{Libations}
\lhead{\textbf{Mike-A-Rita}}
\chead{}
\title{Mike-A-Rita}

\begin{document}
This is the recipe for the classic Mike-A-Rita. It goes nicely with the salsa. Hell, it goes nicely
with just about anything. Because tequila. Makes one.

\bigskip

\bigskip

\textbf{Ingredients}

\begin{itemize}
      \item 2 parts Don Julio Anejo
      \item 1 part Cointreau
      \item \sfrac{3}{4} part freeze squeezed lime
      \item \sfrac{1}{4} part freeze squeezed lemon
      \item 1 part simple syrup
\end{itemize}

\bigskip

\textbf{Directions}

Measure it all out. Pour it over ice. This makes one Mike-A-Rita. Enjoy. Just in case you're
wondering, I don't put any salt on the rim. It's a waste of good salt. However, sometimes
it's actually kinda tasty to put a pinch of salt \textit{in} the mike-a-rita before you
pour it. It's a personal preference and depends on whether the limes are really fresh
and sour. Your call. I could go either way.

% I suppose pictures could go here

\end{document}
