\documentclass{article}
\usepackage{lmodern} % Fix the font size warnings
\usepackage{fancyhdr} % Required for custom headers
\usepackage{lastpage} % Required to determine the last page for the footer
\usepackage{extramarks} % Required for headers and footers
\usepackage{graphicx} % Required to insert images
\usepackage{xfrac} % Nice fractions
\usepackage{amsmath}

\usepackage{multicol}

% Margins
\topmargin=-0.5in
\evensidemargin=0in
\oddsidemargin=-0.5in
\textwidth=7.5in
\textheight=9.0in
\headsep=0.25in 

% paragraphs
\usepackage{parskip}

\pagestyle{fancy}

\rhead{Bread} % Top right header
\lhead{\textbf{Whole Wheat Bread}}
\chead{}
\title{Whole Wheat Bread}

\begin{document}

This one of those recipes that took years to develop. I lost count of how many
loaves of bread I made that just didn't quite work out. Each time I made this I
would slightly change the recipe until one day I finally got that exact rise
and texture that I wanted.

This bread is great sliced thick and toasted for breakfast. I prefer to put honey
or peanut butter on mine. The girls prefer a lot of butter.

\bigskip

\bigskip

\begin{multicols}{2}
    \textbf{Ingredients}

    \begin{itemize}
        \item 2 cups warm water (100 to 110 $^{\circ}$F)
        \item 3 tablespoons molasses
        \item 1 packet active dry yeast (\sfrac{1}{4} ounce or 1 tablespoon)
        \item 3 cups all-purpose flour, divided
        \item 2\sfrac{1}{2} cups whole wheat flour
        \item 1 cup uncooked regular oats
        \item 2 tablespoons vital wheat gluten
        \item 1 tablespoon salt
        \item \sfrac{1}{4} cup honey
        \item 3 tablespoons olive oil
        \item 6 tablespoons all-purpose flour (four dusting)
        \item vegetable shortening
    \end{itemize}

    \bigskip

    \textbf{Directions}

    Combine first three ingredients in a 2-cup glass measuring cup; let yeast
    mixture stand for five minutes.
    \medskip

    Combine 2 cups all-purpose flour, whole wheat flour, oats, wheat gluten, and salt in a large bowl.
    \medskip

    Beat yeast mixture, 1 cup all-purpose flour, honey, and olive oil at medium speed with
    a heavy-duty electric stand mixer until well blended. Gradually add whole wheat flour
    mixture, one cup at a time, beating at a low speed until a soft dough forms.
    \medskip

    Switch to your bread hook and continue kneading the dough on a low (I use setting 2)
    speed for 8 minutes until the dough is smooth and elastic. Place the dough into a
    large bowl that's been lightly oiled or sprayed with a vegetable cooking spray.
    \medskip

    Cover bowl of dough with plastic wrap, and let rise in a warm place (85 $^{\circ}$F), free from
    drafts, for at least one hour or until doubled in size.
    \medskip

    Punch down dough and divide in half. Roll each portion into a 13x8-inch rectangle on a
    lightly floured surface. Roll up each dough rectangle, starting at the short side,
    jelly-roll fashion; pinch ends to seal. Place loaves, seam sides down, into 2
    (8\sfrac{1}{2}$\times$4\sfrac{1}{2}-inch) loaf pans greased with the vegtable shortening.
    \medskip

    Cover loosely with plastic wrap and let rise in a warm place (85 $^{\circ}$F), for another
    45 to 60 minutes, or until almost doubled in size. Remove and discard the plastic wrap.
    \medskip

    Bake at 350 $^{\circ}$F for 30 to 35 minutes or until loaves sound hollow when tapped on the bottom.
    Cool in pans on wire racks for 10 minutes. Remove loaves from pans and cool on the wire racks.
    \medskip

\end{multicols}

Note: If you don't have a heavy-duty stand mixer you may mix or beat the dough by
hand with a wooden spoon. You will also want to turn the dough out onto a floured
surface and knead by hand for 9 minutes. Keep the work surface lightly floured as
the dough will be quite sticky.
\medskip

% I suppose pictures could go here

\end{document}
