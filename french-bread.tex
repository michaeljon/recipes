\documentclass{article}
\usepackage{lmodern} % Fix the font size warnings
\usepackage{fancyhdr} % Required for custom headers
\usepackage{lastpage} % Required to determine the last page for the footer
\usepackage{extramarks} % Required for headers and footers
\usepackage{graphicx} % Required to insert images
\usepackage{xfrac} % Nice fractions
\usepackage{amsmath}

\usepackage{multicol}

% Margins
\topmargin=-0.5in
\evensidemargin=0in
\oddsidemargin=-0.5in
\textwidth=7.5in
\textheight=9.0in
\headsep=0.25in 

% paragraphs
\usepackage{parskip}

\pagestyle{fancy}

\rhead{Breads}
\lhead{\textbf{French Bread}}
\chead{}
\title{French Bread}

\begin{document}
This is a very basic French bread recipe which can be used for boules or baguettes. It's
deceptively simple and is something that can be baked without the recipe after a while. We
prefer to make this the same day we make Chicken Soup because there's nothing better than
fresh bread and hot soup.

\bigskip

\textbf{Ingredients}

\begin{multicols}{2}
      \begin{itemize}
            \item 2 cups warm water (105 to 115 degrees)
            \item 1\sfrac{1}{2} tablespoons active dry yeast
            \item 1 tablespoon sugar
            \item 3 cups bread flour
            \item 1 tablespoon salt
            \item About 3 cups white all-purpose flour
            \item Cornmeal for spinkling on the parchment
            \item 1 large egg beaten with 2 teaspoons water, for glazing
      \end{itemize}
\end{multicols}

\textbf{Directions}

\begin{enumerate}
      \item In the bowl of a heavy-duty stand mixer fitted with the paddle
            attachment pour the warm water, stir in the sugar, and sprinkle the
            yeast over the top. Let stand at room temperature until dissolved
            and foamy. About 10 minutes.
      \item Add 2 cups of the bread flour and the salt. Beat hard until smooth,
            about 3 minutes. Add the remaining 1 cup bread flour and most of the
            all-purpose flour, about \sfrac{1}{2} cup at a time, until you get a
            rough dough that pulls away from the sides.
      \item If kneading by hand, turn dough out onto a lightly floured silicon
            surface and knead until soft, silky, and resilient to tearing, about
            5 to 8 minutes. If necessary dust the surface with flour, 1 tablespoon
            at a time, to keep from sticking. If kneading by machine, insert the
            dough hook and run on low (2) speed for 2 to 4 minutes.
      \item Place the dough in a \textit{lightly} greased deep bowl turning once
            to coat the ball. Cover lightly with plastic wrap and allow to rise
            in a cool area until tripled in size, 1\sfrac{1}{2} to 2 hours. If you
            have time for a second rise, push down the dough and allow to rise
            for 1 more hour. You can let the dough rise in the refrigerator overnight.
      \item Gently deflate the dough and turn out onto a lightly floured silicon surface.
            Line a baking sheet with parchment and spinkle lightly with the cornmeal.
            Divide the dough into three equal parts (works out to about 15\sfrac{1}{2}
            ounces each). Knead in a little more flour now if the dough seems particularly
            sticky. Shape the dough into tight round balls for boules or flatten each
            portion into a rectangle for baguettes. Roll each rectangle up tightly to
            form a long baguette shape and roll to adjust for length.
\end{enumerate}

\textbf{Baking}

There are two ways to bake this bread. With the quick method, immediately after forming
the loaves, slash the tops diagonally no more than \sfrac{1}{4} inch deep and brush
the entire loaf with the glaze. Place in a \textit{cold} oven on the middle or lower rack.
Turn the oven to 400 degrees and bake for 30 to 35 minutes.

With the traditional method, preheat a baking stone to 450 degrees for at least 20 minutes.
If not using a baking stone preheat the oven to 400 degrees. Cover the loaves lightly with
plastic wrap and allow to rise until doubled in size, about 30 to 40 minutes. Remove the
plastic wrap, diagonally slash the loaves no more than \sfrac{1}{4} inch deep, and brush
with the glaze. If using a stone turn the oven temperature down to 400 degrees and place
the parchment on the stone, otherwise put the baking sheets on the middle or lower rack and
bake for 30 to 35 minutes.


\end{document}
