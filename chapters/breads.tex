\chapter{Breads}

\ChapterIntro{
    \lettrine{T}{his is the} chapter intro. \lipsum[1-3]
}

%
%
% French bread
%
%
\newpage

\RecipeNameAndYield{Name=French Bread}

\RecipeStory{\lettrine{T}{his is a very basic} French bread recipe which can be used for boules or baguettes. It's
    deceptively simple and is something that can be baked without the recipe after a while. We
    prefer to make this the same day we make Chicken Soup because there's nothing better than
    fresh bread and hot soup.}

\begin{IngredientsAndSteps}
    \ListIngredientsAndSteps
    {
        2 cups warm water (105 to 115 degrees)

        1\fr1/2 tablespoons active dry yeast

        1 tablespoon sugar

        3 cups bread flour

        1 tablespoon salt

        About 3 cups white all-purpose flour

        Cornmeal for spinkling on the parchment

        1 large egg beaten with 2 teaspoons water, for glazing
    }
    {
        In the bowl of a heavy-duty stand mixer fitted with the paddle
        attachment pour the warm water, stir in the sugar, and sprinkle the
        yeast over the top. Let stand at room temperature until dissolved
        and foamy. About 10 minutes.

        Add 2 cups of the bread flour and the salt. Beat hard until smooth,
        about 3 minutes. Add the remaining 1 cup bread flour and most of the
        all-purpose flour, about \fr1/2 cup at a time, until you get a
        rough dough that pulls away from the sides.

        If kneading by hand, turn dough out onto a lightly floured silicon
        surface and knead until soft, silky, and resilient to tearing, about
        5 to 8 minutes. If necessary dust the surface with flour, 1 tablespoon
        at a time, to keep from sticking. If kneading by machine, insert the
        dough hook and run on low (2) speed for 2 to 4 minutes.

        Place the dough in a \textit{lightly} greased deep bowl turning once
        to coat the ball. Cover lightly with plastic wrap and allow to rise
        in a cool area until tripled in size, 1\fr1/2 to 2 hours. If you
        have time for a second rise, push down the dough and allow to rise
        for 1 more hour. You can let the dough rise in the refrigerator overnight.

        Gently deflate the dough and turn out onto a lightly floured silicon surface.
        Line a baking sheet with parchment and spinkle lightly with the cornmeal.
        Divide the dough into three equal parts (works out to about 15\fr1/2
        ounces each). Knead in a little more flour now if the dough seems particularly
        sticky. Shape the dough into tight round balls for boules or flatten each
        portion into a rectangle for baguettes. Roll each rectangle up tightly to
        form a long baguette shape and roll to adjust for length.
    }
\end{IngredientsAndSteps}

\begin{IngredientsAndSteps}
    \ListIngredientsAndSteps
    {}
    {
        There are two ways to bake this bread. With the quick method, immediately after forming
        the loaves, slash the tops diagonally no more than \fr1/4 inch deep and brush
        the entire loaf with the glaze. Place in a \textit{cold} oven on the middle or lower rack.
        Turn the oven to 400 degrees and bake for 30 to 35 minutes.

        With the traditional method, preheat a baking stone to 450 degrees for at least 20 minutes.
        If not using a baking stone preheat the oven to 400 degrees. Cover the loaves lightly with
        plastic wrap and allow to rise until doubled in size, about 30 to 40 minutes. Remove the
        plastic wrap, diagonally slash the loaves no more than \fr1/4 inch deep, and brush
        with the glaze. If using a stone turn the oven temperature down to 400 degrees and place
        the parchment on the stone, otherwise put the baking sheets on the middle or lower rack and
        bake for 30 to 35 minutes.
    }
\end{IngredientsAndSteps}

%
%
% Whole Wheat Bread
%
%

\newpage

\RecipeNameAndYield{Name=Whole Wheat Bread}

\RecipeStory{\lettrine{T}{his one of those recipes} that took years to develop. I lost count
    of how many loaves of bread I made that just didn't quite work out. Each time I made this I
    would slightly change the recipe until one day I finally got that exact rise
    and texture that I wanted.

    This bread is great sliced thick and toasted for breakfast. I prefer to put honey
    or peanut butter on mine. The girls prefer a lot of butter.}

\begin{IngredientsAndSteps}
    \ListIngredientsAndSteps
    {
        2 cups warm water (100 to 110 degrees)

        3 tablespoons molasses

        1 packet active dry yeast (\fr1/4 ounce or 1 tablespoon)

        3 cups all-purpose flour, divided

        2\fr1/2 cups whole wheat flour

        1 cup uncooked regular oats

        2 tablespoons vital wheat gluten

        1 tablespoon salt

        \fr1/4 cup honey

        3 tablespoons olive oil

        6 tablespoons all-purpose flour (four dusting)

        vegetable shortening
    }
    {
        Combine first three ingredients in a 2-cup glass measuring cup; let yeast
        mixture stand for five minutes.

        Combine 2 cups all-purpose flour, whole wheat flour, oats, wheat gluten, and salt in a large bowl.

        Beat yeast mixture, 1 cup all-purpose flour, honey, and olive oil at medium speed with
        a heavy-duty electric stand mixer until well blended. Gradually add whole wheat flour
        mixture, one cup at a time, beating at a low speed until a soft dough forms.

        Switch to your bread hook and continue kneading the dough on a low (I use setting 2)
        speed for 8 minutes until the dough is smooth and elastic. Place the dough into a
        large bowl that's been lightly oiled or sprayed with a vegetable cooking spray.

        Cover bowl of dough with plastic wrap, and let rise in a warm place (85 degrees), free from
        drafts, for at least one hour or until doubled in size.

        Punch down dough and divide in half. Roll each portion into a \Inch{\AxB{8}{13}} rectangle on a
        lightly floured surface. Roll up each dough rectangle, starting at the short side,
        jelly-roll fashion; pinch ends to seal. Place loaves, seam sides down, into
        2 \Inch{\AxB{8\sfrac{1}{2}}{4\sfrac{1}{2}}}
        loaf pans greased with the vegtable shortening.

        Cover loosely with plastic wrap and let rise in a warm place (85 degrees), for another
        45 to 60 minutes, or until almost doubled in size. Remove and discard the plastic wrap.

        Bake at 350 degrees for 30 to 35 minutes or until loaves sound hollow when tapped on the bottom.
        Cool in pans on wire racks for 10 minutes. Remove loaves from pans and cool on the wire racks.
    }
\end{IngredientsAndSteps}

\begin{ChefNotes}
    {If you don't have a heavy-duty stand mixer you may mix or beat the dough by
        hand with a wooden spoon. You will also want to turn the dough out onto a floured
        surface and knead by hand for 9 minutes. Keep the work surface lightly floured as
        the dough will be quite sticky.}
\end{ChefNotes}
