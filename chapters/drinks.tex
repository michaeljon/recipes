\chapter{Drinks}

\ChapterIntro{
      \lettrine{T}{his is the} chapter intro. \lipsum[1-3]
}

%
%
% Grapefruit margarita
%
%
\newpage

\RecipeNameAndYield{Name=Grapefruit margarita, Yield=Makes 2 drinks}

\RecipeStory{\lettrine{T}{his is not your typical} margarita, nor is it a typical grapefruit margarita. This one is a bit spicy.}

\begin{IngredientsAndSteps}
      \ListIngredientsAndSteps
      {
            4 oz silver tequila

            1 oz Cointreau

            1 oz lime juice

            1 oz lemon juice

            4 oz grapefruit juice

            2 oz Habanero Simple Syrup
      }
      {
            Simple - mix it together, stir with ice, strain into two martini-sized glasses.
      }
\end{IngredientsAndSteps}

%
%
% Habanero Simple Syrup
%
%
\newpage

\RecipeNameAndYield{Name=Habanero Simple Syrup, XRefLabel=habanero-syrup}

\RecipeStory{\lettrine{T}{his simple syrup is used} as the basis for both the grapefruit and mango margaritas. It can go in
      lots of other drinks too. Of course it can. Just make sure you don't get the habaneros in your
      eyes. Maybe wear gloves?}

\begin{IngredientsAndSteps}
      \ListIngredientsAndSteps
      {
            2 cups warm water

            1 granulated sugar (superfine is even better)

            2 habanero peppers, sliced
      }
      {
            Add all ingredients to a 1 quart mason jar. Shake until sugar is disolved. Refrigerate
            overnight to allow the syrup to chill and for the habaneros to impart their general
            goodness into the mix. After 24 - 48 hours in the 'fridge remove the habaneros with a
            slotted spoon.
      }
\end{IngredientsAndSteps}

%
%
% Dad's Sore Throat Remedy
%
%
\newpage

\RecipeNameAndYield{Name=Dad's Sore Throat Remedy}

\RecipeStory{\lettrine{A} {classic, good enough to just} drink on those really chilly nights. But, this one
      is special because it makes sore throats feel better. Officially, that makes this a medicine and medicine is good for you.}

\begin{IngredientsAndSteps}
      \ListIngredientsAndSteps
      {
            \fr3/4 cup very hot water

            \fr1/4 cup bourbon whisky (and don't use the cheap stuff, life's too short)

            2 - 3 tablespoons clover honey

            juice of \fr1/2 fresh lemon

            one lemon slice

            one cinnamon stick

            three cloves
      }
      {
            Mix all the liquid ingredients in a large mug or a really heavy glass.

            Garnish with the cinnamon stick, cloves, and the lemon slice. Serve hot. Serve twice.
      }
\end{IngredientsAndSteps}

%
%
% Mango Habanero Margarita
%
%
\newpage

\RecipeNameAndYield{Name=Mango Habanero Margarita}

\RecipeStory{\lettrine{I}{f you have some} very hot habanero pepper bitters then use those, otherwise use the
      recipe for habanero syrup. This makes a mess of margaritas. Well, two probably. YMMV
}

\begin{IngredientsAndSteps}
      \ListIngredientsAndSteps
      {
            8 oz mango puree / juice

            2 oz Habanero Simple Syrup

            5 oz white tequila (and not the cheap shit)

            1.5 oz triple sec / Cointreau

            3 oz fresh squeezed lime juice
      }
      {
            If everything starts out cold then just mix this in a pitcher and pour it. We don't drink
            it with or over ice. That's no good. Why bother watering down something this tasty.
      }
\end{IngredientsAndSteps}

%
%
% Mike-A-Rita
%
%
\newpage

\RecipeNameAndYield{Name=Mike-A-Rita, Yield=One. Only One.}

\RecipeStory{\lettrine{T}{his is the recipe} for the classic Mike-A-Rita. It goes nicely with the salsa. Hell, it goes nicely
      with just about anything. Because tequila. Makes one.}

\begin{IngredientsAndSteps}
      \ListIngredientsAndSteps
      {
            2 parts Don Julio Anejo

            1 part Cointreau

            \fr3/4 part freeze squeezed lime

            \fr1/4 part freeze squeezed lemon

            1 part simple syrup
      }
      {
            Measure it all out. Pour it over ice. This makes one Mike-A-Rita. Enjoy. Just in case you're
            wondering, I don't put any salt on the rim. It's a waste of good salt. However, sometimes
            it's actually kinda tasty to put a pinch of salt \textit{in} the Mike-A-Rita before you
            pour it. It's a personal preference and depends on whether the limes are really fresh
            and sour. Your call. I could go either way.
      }
\end{IngredientsAndSteps}
