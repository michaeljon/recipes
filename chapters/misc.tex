%!TEX root = ../miller-cooks.tex
%!TEX encoding = UTF-8 Unicode
%!TEX spellcheck = en-US

\chapter{All the other stuff}

\ChapterIntro{
    \lettrine{T}{his is the} chapter intro. \lipsum[1-3]
}

%
%
% salsa
%
%
\newpage

\RecipeNameAndYield{Name=Roasted tomato salsa, Yield=About 12 pints, XRefLabel={misc:salsa}}

\RecipeStory{\lettrine{I}{'ve played around with this} recipe a few times and stumbled on making this
    with canned, whole tomatoes by accident. Follow this recipe, as is, for a
    single batch (makes about four cups), and enjoy immediately.
}

\begin{IngredientsAndSteps}
    \ListIngredientsAndSteps
    {
        2 large cans whole tomatoes

        1 medium yellow onion

        6 or so garlic cloves, unpeeled

        3 large jalapeno peppers

        1 or 2 serano peppers (depending on heat factor)

        1 cup cilantro

        1 \tsp salt

        \fr1/2 \tsp black pepper

        1 \tsp cumin

        1 \tsp coriander

        1 \tsp cider vinegar

        1 lime, juiced
    }
    {
        Slice the jalapeno peppers lengthwise and cut off their tops. Peel and
        quarter the onion. Toss 5 halfs of the jalapeno with the onion in a
        little olive oil and spread on a jelly roll pan. Put the garlic on the
        pan with the other vegetables.

        Drain and crush by hand the tomatoes (watch out, the juice flies all
        over the place) and retain all juices. Put the crushed tomatoes on the
        jelly roll pan. They will cover about half of the pan and the other
        vegetables will cover the other half.

        Roast the vegetables under the broiler for about 8 minutes. Turn the
        peppers and onions over and roast for another 5 minutes. Remove the
        garlic and place in a bowl to cool. Return the pan to the broiler until
        some of the vegetables start to turn black.

        Remove everything from the broiler and let it all cool. When it's cool,
        put all of the roasted vegetables, the remaining \fr1/2 jalapeno, the salt,
        pepper, cumin, coriander, and vinegar into a food processor and pulse a
        few times. Add the lime juice and all of the remaining tomato juice to the
        food processor and pulse until desired texture.

        Taste and adjust seasoning as necessary. If the peppers are too hot
        consider pulsing in about 1 \tsp of white sugar. Remove from processor
        and enjoy. If you want more heat add in one of the serano peppers and taste
        again. You might want the second.

        This is also a good time to experiment a little. Feel free to add more heat,
        or more (raw) garlic, or, well, whatever you want from a salsa.
    }
\end{IngredientsAndSteps}

\begin{ChefNotes}
    {You can quadruple the above ingredients to make enough salsa to can 12 pint
        jars. When I do this I roast the jalapeno, onion, and garlic in one batch,
        then the tomatoes on their own in two batches.

        To can them, prepare your canning jars (wash thoroughly with warm soapy
        water), and add \fr1/4 \tsp of powdered citric acid to each jar.

        Fill the jars to within \Inch{\sfrac{1}{2}} of the top, clean the rim, and screw
        on the ring. Place in boiling water so that the tops are covered by at
        least one inch of water and boil for 35 minutes. Carefully remove the
        jars and place on a dry towel to cool. The lids should pop over the next
        few hours. If one doesn't seal, well, eat that jar.}
\end{ChefNotes}