\chapter{Breakfasts}

\ChapterIntro{
    \lettrine{T}{his is the} chapter intro. \lipsum[1-3]
}

%
%
% Crepes
%
%
\newpage

\RecipeNameAndYield{Name=Crepes, Yield=About 6 \Inch{12} crepes}

\RecipeStory{\lettrine{T}{...} tbd
}

\begin{IngredientsAndSteps}
    \ListIngredientsAndSteps
    {
    }
    {
    }
\end{IngredientsAndSteps}

%
%
% Scones
%
%
\newpage

\RecipeNameAndYield{Name=Simple scones}

\RecipeStory{\lettrine{T}{his recipe was originally} given to me by Kendall Anderegg right
    after Caitlyn was born. I've tweaked it a bit over the years to make sure that each
    batch is consistently baked (they were a little too big and wanted to spread a lot).
    You can docter this recipe a bit by adding a cup of raisins or dried cranberries after
    you've mixed in the wet ingredients. It's not a thing that happens in our house.

    I prefer a drier Greek yogurt for this recipe. The others have too much moisture and
    tend to make the scones stickier. I also like a really crunchy raw sugar because it
    adds a nice `bite' to the finished scones, but you can use any kind of sugar you want,
    even colored sugars for a holiday flair.

    You can turn these into blueberry (or other) scones. See below.
}

\begin{IngredientsAndSteps}
    \ListIngredientsAndSteps
    {
        1 cup sour cream or plain Greek yogurt

        1 \tsp baking soda

        4 cups all-purpose white flour

        1 cup white sugar

        2 \tsp[s] baking powder

        \fr1/4 \tsp cream of tartar

        1 \tsp salt

        1 cup (2 sticks) butter

        2 \Tbl[s] milk (optional))

        1 egg
    }
    {
        Preheat the oven to 350\Degrees[F] and line two baking sheets with parchment paper.

        In a small bowl, blend the sour cream or yogurt with the baking soda and set aside.

        In a large bowl mix the flour, sugar, salt, baking powder, and cream of tartar.

        Cut in the butter. Stir in the sour cream or yogurt mixture and one egg until just moistened.
        You may need to add up two 2 \Tbl[s] of milk to get the right consistency.

        Turn the dough out onto a lightly floured surface and knead until smooth. Spinkle additional
        flour onto the surface to keep the dough from sticking.

        Form the dough into a 12 inch ``log'', cut it into four equal-sized pieces, and flatten
        each piece into an approximately 4 inch tapered round. Cut each piece into quarters.

        Place the quarters onto the baking sheets. Whisk the remaining egg and milk in a small bowl
        and brush the egg / milk mixture onto each scone. Sprinkle raw sugar over each scone.
    }

    \ListIngredientsAndSteps[Glaze ingredients]
    {
        1 egg

        2 \Tbl[s] milk

        raw sugar
    }
    {}
\end{IngredientsAndSteps}

\BakeUntil{Min=16, Max=18, GBrown=1}
\Attribution{Recipe courtesy of Kendall Anderegg}

\begin{Tip}
    {
        If you want blueberry scones then when you've cut the dough into four pieces, roll each
        one out into an \Inch{\AxB{8}{8}} square, drop blueberries over the top, then roll the
        section. After rolling cut the new (flattened) section (it'll look like a biscotti section)
        into half, then cut each half diagonally to end up with four sections.
    }
\end{Tip}

%
%
% Dad's Waffles
%
%
\newpage

\RecipeNameAndYield{Name=Dad's Waffles, Yield=About 5 waffles}

\RecipeStory{\lettrine{T}{his is a very basic recipe} for waffles. It's also tweakable (fruits, different
    flavors, other stuff). I think this started with a recipe that was included
    with a waffle maker that we purchased, but that recipe was lost long ago.
}

\begin{IngredientsAndSteps}
    \ListIngredientsAndSteps
    {
        1 cup flour

        1\fr1/2 \tsp[s] baking powder

        \fr1/2 \tsp salt

        3 \Tbl[s] sugar

        1 large egg

        3 - 4 \Tbl[s] butter, melted

        1 cup milk

        1 \tsp vanilla
    }
    {
        Combine all dry ingredients in a medium bowl. Melt butter in
        microwave and let cool slightly. Add milk, vanilla, and egg to
        dry ingredients and combine until reasonably smooth. Add butter and stir.
        The batter should be thick, but not unpourable.

        Let mix rest 5 to 10 minutes while the waffle iron is warming up.

        Set waffle iron to somewhere between 4 and 5 and wait until it's warm.
        Scoop \fr1/2 cup waffle batter into iron and bake until done.
    }
\end{IngredientsAndSteps}

