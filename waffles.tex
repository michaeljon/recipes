\documentclass{article}
\usepackage{lmodern} % Fix the font size warnings
\usepackage{fancyhdr} % Required for custom headers
\usepackage{lastpage} % Required to determine the last page for the footer
\usepackage{extramarks} % Required for headers and footers
\usepackage{graphicx} % Required to insert images
\usepackage{xfrac} % Nice fractions
\usepackage{amsmath}

\usepackage{multicol}

% Margins
\topmargin=-0.5in
\evensidemargin=0in
\oddsidemargin=-0.5in
\textwidth=7.5in
\textheight=9.0in
\headsep=0.25in 

% paragraphs
\usepackage{parskip}

\pagestyle{fancy}

\rhead{Miller Family} % Top right header
\lhead{\textbf{ad's Waffles}}
\chead{}
\title{Dad's Waffles}

\begin{document}
This is a very basic recipe for waffles. It's also tweakable (fruits, different
flavors, other stuff). I think this started with a recipe that was included
with a waffle maker that we purchased, but that recipe was lost long ago.

Makes about 5 waffles.

\bigskip
\bigskip

\textbf{Ingredients}

\begin{itemize}
    \item 1 cup flour
    \item 1\sfrac{1}{2} teaspoons baking powder
    \item \sfrac{1}{2} teaspoon salt
    \item 3 tablespoons sugar
    \item 1 large egg
    \item 3 - 4 tablespoons butter, melted
    \item 1 cup milk
    \item 1 teaspoon vanilla
\end{itemize}

\bigskip

\textbf{Directions}

Combine all dry ingredients in a medium bowl. Melt butter in
microwave and let cool slightly. Add milk, vanilla, and egg to
dry ingredients and combine until reasonably smooth. Add butter and stir.
The batter should be thick, but not unpourable.

Let mix rest 5 to 10 minutes while the waffle iron is warming up.

Set waffle iron to somewhere between 4 and 5 and wait until it's warm.
Scoop \sfrac{1}{2} cup waffle batter into iron and bake until done.

% I suppose pictures could go here

\end{document}
