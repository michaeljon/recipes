\documentclass{article}
\usepackage{lmodern} % Fix the font size warnings
\usepackage{fancyhdr} % Required for custom headers
\usepackage{lastpage} % Required to determine the last page for the footer
\usepackage{extramarks} % Required for headers and footers
\usepackage{graphicx} % Required to insert images
\usepackage{xfrac} % Nice fractions
\usepackage{amsmath}
\usepackage{mathabx}

\usepackage{multicol}

% Margins
\topmargin=-0.5in
\evensidemargin=0in
\oddsidemargin=-0.5in
\textwidth=7.5in
\textheight=9.0in
\headsep=0.25in 

% paragraphs
\usepackage{parskip}

\pagestyle{fancy}

\rhead{Breakfasts}
\lhead{\textbf{Simple scones}}
\chead{}
\title{Simple scones}

\begin{document}
This recipe was originally given to me by Kendall Anderegg right after Caitlyn was
born. I've tweaked it a bit over the years to make sure that each batch is consistently
baked (they were a little too big and wanted to spread a lot). You can docter this
recipe a bit by adding a cup of raisins or dried cranberries after you've mixed in
the wet ingredients. It's not a thing that happens in our house.

I prefer a drier Greek yogurt for this recipe. The others have too much moisture and
tend to make the scones stickier. I also like a really crunchy raw sugar because it
adds a nice `bite' to the finished scones, but you can use any kind of sugar you want,
even colored sugars for a holiday flair.

You can turn these into blueberry (or other) scones. See below.

\bigskip

\begin{multicols}{2}
      \textbf{Ingredients}

      \begin{itemize}
            \item 1 cup sour cream or plain Greek yogurt
            \item 1 teaspoon baking soda
            \item 4 cups all-purpose white flour
            \item 1 cup white sugar
            \item 2 teaspoons baking powder
            \item \sfrac{1}{4} teaspoon cream of tartar
            \item 1 teaspoon salt
            \item 1 cup (2 sticks) butter
            \item 2 tablespoons milk (optional))
            \item 1 egg
      \end{itemize}

      \columnbreak

      \textbf{Glaze ingredients}
      \begin{itemize}
            \item 1 egg
            \item 2 tablespoons milk
            \item raw sugar
      \end{itemize}

\end{multicols}


\textbf{Directions}

\begin{enumerate}
      \item{Preheat the oven to 350 degrees and line two baking sheets with parchment paper.}
      \item{In a small bowl, blend the sour cream or yogurt with the baking soda and set aside.}
      \item{In a large bowl mix the flour, sugar, salt, baking powder, and cream of tartar.}
      \item{Cut in the butter. Stir in the sour cream or yogurt mixture and one egg until just moistened.
                  You may need to add up two 2 tablespoons of milk to get the right consistency.}
      \item{Turn the dough out onto a lightly floured surface and knead until smooth. Spinkle additional
                  flour onto the surface to keep the dough from sticking.}
      \item{Form the dough into a 12 inch ``log'', cut it into four equal-sized pieces, and flatten
                  each piece into an approximately 4 inch tapered round. Cut each piece into quarters.}
      \item{Place the quarters onto the baking sheets. Whisk the remaining egg and milk in a small bowl
                  and brush the egg / milk mixture onto each scone. Sprinkle raw sugar over each scone.}
\end{enumerate}

\bigskip

Bake for 16 to 18 minutes or until evenly browned.

If you want blueberry scones then when you've cut the dough
into four pieces, roll each one out into an $8\times8$ square,
drop blueberries over the top, then roll the section. After
rolling cut the new (flattened) section (it'll look like a
biscotti section) into half, then cut each half diagonally
to end up with four sections.

% I suppose pictures could go here

\end{document}
