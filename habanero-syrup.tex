\documentclass{article}
\usepackage{lmodern} % Fix the font size warnings
\usepackage{fancyhdr} % Required for custom headers
\usepackage{lastpage} % Required to determine the last page for the footer
\usepackage{extramarks} % Required for headers and footers
\usepackage{graphicx} % Required to insert images
\usepackage{xfrac} % Nice fractions
\usepackage{amsmath}

\usepackage{multicol}

% Margins
\topmargin=-0.5in
\evensidemargin=0in
\oddsidemargin=-0.5in
\textwidth=7.5in
\textheight=9.0in
\headsep=0.25in 

% paragraphs
\usepackage{parskip}

\pagestyle{fancy}

\rhead{Libations} % Top right header
\lhead{\textbf{Habanero Simple Syrup}}
\chead{}
\title{Habanero Simple Syrup}

\begin{document}

This simple syrup is used as the basis for both the grapefruit and mango margaritas. It can go in
lots of other drinks too. Of course it can. Just make sure you don't get the habaneros in your
eyes. Maybe wear gloves?

\textbf{Ingredients}

\begin{itemize}
    \item 2 cups warm water
    \item 1 granulated sugar (superfine is even better)
    \item 2 habanero peppers, sliced
\end{itemize}

\textbf{Directions}


Add all ingredients to a 1 quart mason jar. Shake until sugar is disolved. Refrigerate
overnight to allow the syrup to chill and for the habaneros to impart their general
goodness into the mix. After 24 - 48 hours in the 'fridge remove the habaneros with a
slotted spoon.

% I suppose pictures could go here

\end{document}
