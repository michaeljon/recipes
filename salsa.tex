\documentclass{article}
\usepackage{lmodern} % Fix the font size warnings
\usepackage{fancyhdr} % Required for custom headers
\usepackage{lastpage} % Required to determine the last page for the footer
\usepackage{extramarks} % Required for headers and footers
\usepackage{graphicx} % Required to insert images
\usepackage{xfrac} % Nice fractions
\usepackage{amsmath}

\usepackage{multicol}

% Margins
\topmargin=-0.5in
\evensidemargin=0in
\oddsidemargin=-0.5in
\textwidth=7.5in
\textheight=9.0in
\headsep=0.25in 

% paragraphs
\usepackage{parskip}

\pagestyle{fancy}

\rhead{Other stuff} % Top right header
\lhead{\textbf{Roasted tomato salsa}}
\chead{}
\title{Roasted tomato salsa}

\begin{document}
I've played around with this recipe a few times and stumbled on making this
with canned, whole tomatoes by accident. Follow this recipe, as is, for a
single batch (makes about four cups), and enjoy immediately.

\bigskip

\bigskip

\textbf{Ingredients}

\begin{itemize}
    \item 2 large cans whole tomatoes
    \item 1 medium yellow onion
    \item 6 or so garlic cloves, unpeeled
    \item 3 large jalapeno peppers
    \item 1 or 2 serano peppers (depending on heat factor)
    \item 1 cup cilantro
    \item 1 teaspoon salt
    \item \sfrac{1}{2} teaspoon black pepper
    \item 1 teaspoon cumin
    \item 1 teaspoon coriander
    \item 1 teaspoon cider vinegar
    \item 1 lime, juiced
\end{itemize}

\bigskip

\textbf{Directions}

Slice the jalapeno peppers lengthwise and cut off their tops. Peel and
quarter the onion. Toss 5 halfs of the jalapeno with the onion in a
little olive oil and spread on a jelly roll pan. Put the garlic on the
pan with the other vegetables.
\medskip
Drain and crush by hand the tomatoes (watch out, the juice flies all
over the place) and retain all juices. Put the crushed tomatoes on the
jelly roll pan. They will cover about half of the pan and the other
vegetables will cover the other half.
\medskip
Roast the vegetables under the broiler for about 8 minutes. Turn the
peppers and onions over and roast for another 5 minutes. Remove the
garlic and place in a bowl to cool. Return the pan to the broiler until
some of the vegetables start to turn black.
\medskip
Remove everything from the broiler and let it all cool. When it's cool,
put all of the roasted vegetables, the remaining \sfrac{1}{2} jalapeno, the salt,
pepper, cumin, coriander, and vinegar into a food processor and pulse a
few times. Add the lime juice and all of the remaining tomato juice to the
food processor and pulse until desired texture.
\medskip
Taste and adjust seasoning as necessary. If the peppers are too hot
consider pulsing in about 1 teaspoon of white sugar. Remove from processor
and enjoy. If you want more heat add in one of the serano peppers and taste
again. You might want the second.
\medskip
This is also a good time to experiment a little. Feel free to add more heat,
or more (raw) garlic, or, well, whatever you want from a salsa.
\medskip

\bigskip

\textbf{Canning}

You can quadruple the above ingredients to make enough salsa to can 12 pint
jars. When I do this I roast the jalapeno, onion, and garlic in one batch,
then the tomatoes on their own in two batches.

To can them, prepare your canning jars (wash thoroughly with warm soapy
water), and add \sfrac{1}{4} teaspoon of powdered citric acid to each jar.

Fill the jars to within \sfrac{1}{2} inch of the top, clean the rim, and screw
on the ring. Place in boiling water so that the tops are covered by at
least one inch of water and boil for 35 minutes. Carefully remove the
jars and place on a dry towel to cool. The lids should pop over the next
few hours. If one doesn't seal, well, eat that jar.


% I suppose pictures could go here

\end{document}
