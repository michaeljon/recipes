\documentclass{article}
\usepackage{lmodern} % Fix the font size warnings
\usepackage{fancyhdr} % Required for custom headers
\usepackage{lastpage} % Required to determine the last page for the footer
\usepackage{extramarks} % Required for headers and footers
\usepackage{graphicx} % Required to insert images
\usepackage{xfrac} % Nice fractions
\usepackage{amsmath}

% Margins
\topmargin=-0.5in
\evensidemargin=0in
\oddsidemargin=-0.5in
\textwidth=7.5in
\textheight=9.0in
\headsep=0.25in 

% paragraphs
\usepackage{parskip}

\pagestyle{fancy}

\rhead{Miller Family} % Top right header
\lhead{\textbf{Chicken Lettuce Wraps}}
\chead{}
\title{Chicken Lettuce Wraps}

\begin{document}
This recipe is an adaptation of the PF Chang's Chicken Lettuce Wraps
recipe found on Damn Delicious. We make a pot of white rice and serve
this with butter or iceberg lettuce. It's important to use ``regular''
soy sauce for this recipe. We've found that using low sodium tends
to leave the overall flavor a little too sweet.

\bigskip

\textbf{Ingredients}

\begin{itemize}
      \item 1 tablespoon olive olive
      \item 1 pound ground chicken
      \item 3 cloves garlic, smashed
      \item \sfrac{1}{4} cup hoisin sauce
      \item 2 tablespoons soy sauce (don't use low-sodium)
      \item 1 tablespoon rice wine vinegar
      \item 1\sfrac{1}{2} tablespoons sambal oelek
      \item 2 tablespoons grated fresh ginger
      \item 1 small yellow onion, diced very small
      \item 2 green onions, sliced thin
      \item (optional) 1 (8-ounce) can water chestnuts, drained and diced
      \item kosher salt and pepper to taste (about \sfrac{1}{2} teaspoon each)
\end{itemize}

\bigskip

\textbf{Directions}

\begin{enumerate}
      \item Combine garlic, hoisin, soy, vinegar, and sambal, then side aside.
      \item Heat olive oil in deep skillet over medium high heat. Reduce heat to medium, add ground chicken,
            and cook until browned, about 10 minutes, making sure to crumble the chicken as it cooks.
      \item Add yellow onion, grated ginger, and sauce combination to chicken and allow to cook until
            onions are translucent.
      \item Stir in green onions and cook until tender, about 2 minutes; season with salt
            and pepper, to taste. If using water chestnuts add them at this time and cook
            until they are warmed through.
\end{enumerate}

% I suppose pictures could go here

\end{document}
