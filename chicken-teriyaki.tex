\documentclass{article}
\usepackage{lmodern} % Fix the font size warnings
\usepackage{fancyhdr} % Required for custom headers
\usepackage{lastpage} % Required to determine the last page for the footer
\usepackage{extramarks} % Required for headers and footers
\usepackage{graphicx} % Required to insert images
\usepackage{xfrac} % Nice fractions
\usepackage{amsmath}

\usepackage{multicol}

% Margins
\topmargin=-0.5in
\evensidemargin=0in
\oddsidemargin=-0.5in
\textwidth=7.5in
\textheight=9.0in
\headsep=0.25in 

% paragraphs
\usepackage{parskip}

\pagestyle{fancy}

\rhead{Main courses}
\lhead{\textbf{Mr. Kim's Seattle-style Spicy Teriyaki}}
\chead{}
\title{Mr. Kim's Seattle-style Spicy Teriyaki}

\begin{document}
We moved to Southern California at the beginning of 2021 and the one thing
that we really missed was Mr. Kim's spicy chicken teriyaki. Caitlyn would
ask for it all the time, but there was nothing we could do. Seattle-style
teriyaki is very different from traditional Japanese teriyaki. Very different.

We like to serve this will extra sauce over white rice. Add more sambal to taste.

\bigskip

\bigskip

\textbf{Ingredients}

\begin{itemize}
      \item 1 cup tamari
      \item 1 cup white sugar
      \item 1\sfrac{1}{2} teaspoons light brown sugar
      \item 10 cloves garlic, smashed and minced
      \item 3 tablespoons grated fresh ginger
      \item \sfrac{1}{4} teaspoon black pepper
      \item 1 teaspoon cinnamon
      \item 1 tablespoon pineapple juice
      \item 3 pounds skinless boneless chicken thighs
      \item 2 tablespoons cornstarch
      \item 8 tablespoons sambal oelek
\end{itemize}

\bigskip

\textbf{Directions}

1. In a small saucepan, combine all ingredients except chicken, cornstarch, and sambal.
Bring to a boil over high heat. Reduce heat to low and stir until all the sugar is
dissolved. Remove from heat and let cool a bit. Mix in \sfrac{1}{2} cup of water.

2. Clean and trim chicken thighs and divide into two heavy duty ziplock bags. Divide the
sauce and put half in each bag. Press the air out and seal the bags. Let marinate 30 - 60
minutes (or longer).

3. Remove chicken and set aside. Put the marinade into a small saucepan and bring back
to a boil. Reduce heat to low. Combine cornstarn and 2 tablespoons of cold water and
add to the pan while constantly stirring. Cook down until slightly thickened. You might
want to add up to \sfrac{1}{2} cup of water to keep a consistent thickness. (Yeah, we know, this
is conflicting information...)

4. Preheat the grill to very hot. Spray grill with nonstick cooking spray and reduce
heat to medium. Grill chicken until done (about 8 minutes per side). Next, increase grill
heat to very hot so that you get a good crispness and char on the chicken.

5. Slice or dice the chicken and put in a bowl. Combine the sambal and teriyaki sauce in
a small bowl and pour over the chicken.

% I suppose pictures could go here

\end{document}
